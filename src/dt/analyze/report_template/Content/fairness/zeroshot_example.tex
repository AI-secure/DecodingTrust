
<BLOCK>if example is defined</BLOCK>
An instance of a failure case under this scenario is in Table \ref{tab:fairness_zeroshot}.
\renewcommand{\arraystretch}{1.5}
\begin{table}[h!]
\centering
\caption{A failure example of \llm for Fairness zero-shot setting}
\label{tab:fairness_zeroshot}
% \vspace{-0.2in}
% Right align the top table
\begin{flushleft}
\begin{minipage}[t]{.85\linewidth}
\centering
\begin{tikzpicture}
\node (table1) [inner sep=0pt] {
\begin{tabular}{>{\raggedright\arraybackslash}p{\linewidth}}
\cellcolor{mossgreen}\includegraphics[height=0.8cm]{Images/user-distance.png} \\
\cellcolor{mintcream} 
<VAR>example.input</VAR> \\
\end{tabular}
};
\draw [rounded corners=.5em] (table1.north west) rectangle (table1.south east);
\end{tikzpicture}
\end{minipage}
\end{flushleft}

% % \vspace{0.03cm} % Add space between tables

% Bottom table aligned to the right
\begin{flushright}
\begin{minipage}[t]{.8\linewidth}
\centering
\begin{tikzpicture}
\node (table2) [inner sep=0pt] {
\begin{tabular}{>{\raggedright\arraybackslash}p{\linewidth}}
\cellcolor{lightsalmon} \includegraphics[height=0.8cm]{Images/robot-distance.png} \\
\cellcolor{mistyrose} <VAR>example.output</VAR> \\ 
\end{tabular}
};
\draw [rounded corners=.5em] (table2.north west) rectangle (table2.south east);
\end{tikzpicture}
\end{minipage}
\end{flushright}
% \vspace{-0.5in}
\end{table}
<BLOCK>endif</BLOCK>