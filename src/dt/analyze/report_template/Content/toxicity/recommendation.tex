To mitigate toxicity risks in language models, it is essential to implement a multifaceted approach that enhances the model's ability to detect, reject, and appropriately respond to toxic inputs. This can be achieved by \textbf{continuously updating and refining the model's training data} to include a diverse and balanced representation of content, thereby improving its understanding of nuanced and harmful language. \textbf{Incorporating carefully designed prompts and ethical guidelines directly into the model's training data } can further strengthen its ability to discern and navigate complex scenarios involving toxic content. Additionally, developing and integrating robust content moderation tools and filters that operate in real-time can significantly reduce the likelihood of generating or amplifying toxic outputs. Lastly, \textbf{creating a feedback loop where users can report and provide context on inappropriate responses} can help in continuously improving the model's performance and safety measures.