% \begin{table}[h]
% \vspace{+0.2in}
% \caption{A failure example of \llm for OOD Style}
% \begin{tikzpicture}
% \node (table) [inner sep=0pt] {
% \begin{tabular}{>{\raggedright\arraybackslash}p{0.63\linewidth}>{\raggedright\arraybackslash}p{0.37\linewidth}}
% \cellcolor{mossgreen}\includegraphics[height=0.8cm]{Images/user.png} & \cellcolor{lightsalmon} \includegraphics[height=0.8cm]{Images/robot.png} \\
% \cellcolor{mintcream} You are a helpful assistant.
% For each snippet of text, label the sentiment of the text as positive or negative. The answer should be exact 'positive' or 'negative'.
% In all, a watchful experience  that 'twas nought quite like to the memorable experience & \cellcolor{mistyrose} sentiment: positive</s> \\ 
% \end{tabular}
% };
% \draw [rounded corners=.5em] (table.north west) rectangle (table.south east);
% \end{tikzpicture}
% \label{tab:ood_style}
% \vspace{-0.2in}
% \end{table}

% \begin{table}[h]
% \vspace{+0.2in}
% \centering
% \caption{A failure example of \llm for OOD Style}
% \label{tab:ood_knowledge}

% \begin{minipage}{.6\linewidth}
% \centering
% \begin{tikzpicture}
% \node (table1) [inner sep=0pt] {
% \begin{tabular}{>{\raggedright\arraybackslash}p{\linewidth}}
% \cellcolor{mossgreen}\includegraphics[height=0.8cm]{Images/user.png} \\
% \cellcolor{mintcream} You are a helpful assistant.
% For each snippet of text, label the sentiment of the text as positive or negative. The answer should be exact 'positive' or 'negative'.

% In all, a watchful experience  that 'twas nought quite like to the memorable experience \\ 
% \end{tabular}
% };
% \draw [rounded corners=.5em] (table1.north west) rectangle (table1.south east);
% \end{tikzpicture}
% \end{minipage}%
% \begin{minipage}{.4\linewidth}
% \centering
% \begin{tikzpicture}
% \node (table2) [inner sep=0pt] {
% \begin{tabular}{>{\raggedright\arraybackslash}p{\linewidth}}
% \cellcolor{lightsalmon} \includegraphics[height=0.8cm]{Images/robot.png} \\
% \cellcolor{mistyrose}  sentiment: positive</s> \\ 
% \end{tabular}
% };
% \draw [rounded corners=.5em] (table2.north west) rectangle (table2.south east);
% \end{tikzpicture}
% \end{minipage}
% \end{table}
<BLOCK>if example is defined</BLOCK>
An instance of a failure case under this scenario is in Table \ref{tab:ood_style}.
\renewcommand{\arraystretch}{1.5}
\begin{table}[h!]
\centering
\caption{A failure example of \llm for OOD Style}
\label{tab:ood_style}
% \vspace{-0.2in}
% Right align the top table
\begin{flushleft}
\begin{minipage}[t]{.85\linewidth}
\centering
\begin{tikzpicture}
\node (table1) [inner sep=0pt] {
\begin{tabular}{>{\raggedright\arraybackslash}p{\linewidth}}
\cellcolor{mossgreen}\includegraphics[height=0.8cm]{Images/user-distance.png} \\
\cellcolor{mintcream} <VAR>example.input</VAR> \\
\end{tabular}
};
\draw [rounded corners=.5em] (table1.north west) rectangle (table1.south east);
\end{tikzpicture}
\end{minipage}
\end{flushleft}

\begin{flushright}
\begin{minipage}[t]{.8\linewidth}
\centering
\begin{tikzpicture}
\node (table2) [inner sep=0pt] {
\begin{tabular}{>{\raggedright\arraybackslash}p{\linewidth}}
\cellcolor{lightsalmon} \includegraphics[height=0.8cm]{Images/robot-distance.png} \\
\cellcolor{mistyrose} <VAR>example.output</VAR> \\ 
\end{tabular}
};
\draw [rounded corners=.5em] (table2.north west) rectangle (table2.south east);
\end{tikzpicture}
\end{minipage}
\end{flushright}
% \vspace{-0.3in}
\end{table}
<BLOCK>endif</BLOCK>