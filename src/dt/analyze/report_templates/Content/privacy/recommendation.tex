Given the privacy risks associated with the model's performance, 
% particularly in accurately handling sensitive information and understanding privacy contexts,
we recommend the following actions:
\textbf{1. Enhanced Data Scrutiny:} Implement rigorous data screening and sanitization processes during pretraining to mitigate the risk of inadvertently memorizing sensitive information, especially from datasets like the Enron email dataset. This involves thorough examination and removal of potentially compromising data points. 
\textbf{2. Robust PII Detection Mechanisms:} Strengthen the model's ability to detect and prevent Personally Identifiable Information (PII) leakage during inference by integrating advanced detection mechanisms. This may involve incorporating state-of-the-art algorithms for PII detection and implementing strict rules for handling and redacting sensitive information in outputs.
\textbf{3. Privacy-Aware Conversation Modeling:} Enhance the model's understanding of privacy contexts during conversations by incorporating instruction tuning with diverse training examples involving privacy-related words and events. This can help the model better grasp the nuances of privacy discussions and appropriately handle sensitive topics to minimize the risk of inadvertent information disclosure.
By implementing these recommendations, the model may reduce its privacy risk profile and enhance its capability to safeguard sensitive information effectively.