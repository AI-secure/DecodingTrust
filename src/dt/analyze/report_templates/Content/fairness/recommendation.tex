To mitigate the bias (unfairness) risks of the model predictions, we recommend the following actions: 
\textbf{1. Diversify Training Data:} Ensure the training dataset encompasses a wide range of demographics, including balanced representations of different genders, races, ages, and other relevant characteristics. This diversity helps the model learn from a broader spectrum of examples, reducing the risk of perpetuating biases present in more homogeneous datasets.
\textbf{2. Bias Detection and Correction Techniques:} Implement bias detection methodologies to identify and measure biases within model predictions. Once detected, employ correction techniques, such as re-weighting the training data, modifying the model's objective function to penalize biased predictions, or using post-processing adjustments to balance the output.
\textbf{3. Transparency and Explainability:} Enhance the transparency and explainability of the model. Understanding how and why a model makes certain predictions is crucial for identifying underlying biases. This can involve techniques like feature importance analysis, model-agnostic explainability tools, or logit lens techniques for the track of generation dynamics.